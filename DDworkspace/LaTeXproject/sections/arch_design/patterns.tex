\subsection{Architectural styles}
In the overview section (2.1) there is a description of how a \textbf{three-tier client-server} architecture has been used to design the system. This choice favors, first of all, a greater decoupling of the system, increasing reusability, scalability and
flexibility. 

Regarding the transport protocol adopted, our choice is \textbf{HTTPS}. It's the secure version of HTTP, which is the primary protocol used to send data between the web browser and a website (like the web app used by Employees and the Store Manager). HTTPS is encrypted to increase the security of data transfer. This is particularly important when users transmit sensitive data, such as by logging into their account.

For the communication between the application and the system, our choice is a \textbf{REST} architecture. The goal was to reduce the coupling between client and server components. It fits very well for the scope because the goal is to update the server-side regularly, without touching the client software. This choice introduces the following constraints:
\begin{itemize}
    \item Layered Architecture
    \item Client-Server
    \item Stateless system
    \item Cacheable
\end{itemize}
All these constraints are respected by the choices previously made.

\subsection{Patterns}
    Selected design patterns:
    \begin{itemize}
        \item \textbf{Facade pattern} simplifies a complicated system by providing a single interface to a set of interfaces within a subsystem. It is used for some components designed in the component diagram in section 2.2.
    \end{itemize}
    Recommended architectural patterns for implementation:
    \begin{itemize}
        \item \textbf{Model-View-Controller pattern} is an architectural pattern that divides a given software application into three interconnected parts. This is done to separate internal representations of information from the ways information is presented to and accepted by the user. MVC is recommended because it is the best design pattern to structure our three-tier client-server architecture presented in the previous sections and it is one of the most common and effective ways to avoid a dangerous level of coupling between the various parts of the whole system.
    \end{itemize}
    Recommended design patterns for implementation:
    \begin{itemize}
        \item \textbf{Observer pattern} allows you to notify one or more objects about changes in the state of other objects within the system. It is recommended because it is essential to the previously recommended MVC model and is also useful in other CLup system applications.
        \item \textbf{Visitor pattern} is a way of separating an algorithm from an object structure on which it operates. A practical result of this separation is the ability to add new operations to existing object structures without modifying the structures. It decoupling the operations from the object structure. This pattern is useful for the MVC application too.
        \item \textbf{Factory pattern} is a creational pattern that uses factory methods to deal with the problem of creating objects without having to specify the exact class of the object that will be created. It is particularly useful if applied in combination with the MVC pattern.
    \end{itemize}
